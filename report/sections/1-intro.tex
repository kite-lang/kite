%%% Local Variables:
%%% mode: latex
%%% TeX-master: "../report"
%%% End:
TODO: 1-2 pages
\subsection{Programming languages and compilers in general}
TODO: Give a short description of a compiler.


Programming languages are the means for programmers to express what the machine should compute, or in other words: What a program should do.
The two main divisions between programming languages are likely whether they are High- or Low-level and which `programming paradigm' they follow (in the next section these paradigms will be shortly described).

The low-level languages are close to what the computer `natively` understand, which is machine code/instructions, but are harder for humans to read and write. Examples hereof are FOTRAN and COBOL. Usually high-level languages are used for faster development, better readability and maintenance, among other things. Thus it is of general interest... SOMETHING SOMETHING


TODO: Describe paradigms in the next subsection

A programming language is implemented via a compiler

\subsection{Programming paradigms}
Through the history of computer science, many different approaches to the design and implementation of programming languages have been taken. Most languages can be categorized as being either in the imperative or declarative style. Code written in imperative style is more explicit in \emph{how} a problem is solved, where declarative languages tend to describe \emph{what} is to be computed. There exists multiple language paradigms that fall under one or both of these categories.

A very popular paradigm is \textbf{object-oriented} programming (including \code{C++} and \code{Java}), in which data and algorithms are structured into objects that contain methods that can change the state of data in objects.

Another imperative approach is \textbf{procedural} languages like \code{C}, which is somewhat simpler and is characterized by subroutines, scoped variables and some sort of modules.

A very different style is the highly declarative \textbf{logic} programming, in which execution order and control-flow does not exist, but programs consist of logic clauses that define relationships and implications.

\subsection{The problem}
We are going to implement a compiler for a functional language named Kite. 

In short, we want to make our own functional programming language, Kite, by implementing a compiler.


Kite will feature the basic elements of most functional languages (described above), and a few additional features:

\begin{itemize}

\item Static typing and thus typechecking
\item Closures
\item If-expressions (not to be confused with if-statements)

\end{itemize}

\subsection{The structure of the report}
TODO
