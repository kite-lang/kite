%%% Local Variables:
%%% mode: latex
%%% TeX-master: "../report"
%%% End:

(TODO: Restatement of problem)
We have fulfilled our initial goal of gaining SUBSTANTIAL(something less grandiose) knowledge of compilers and functional languages in general. This was achieved through the process of planning, designing and implementing the Kite programming language.


(TODO: Summary of key points
... eh hvor meget skal der lige staa? Paa en maade er sidste saetning i ovenstaaende et summary (planned, designed and implemented). Skal der staa mere?)


TODO: Concluding statement - recommendations

In conclusion, we are glad we achieved a much better understanding of compilers and in functional languages. (in regards of) Compilers, these are an essential part of computers and as aspiring software engineers, we ought we have a (somewhat?) deep knowledge about the different aspects of computing (in order to perform our best?). As of functional languages, we believe these will play a larger rule in the future, mainly due to the inherent ease of maintainability and concurrency. Concurrency is an important aspect of the future as the clock frequency of CPU are unlikely to increase with silicon based chips, but the transistor and core count is fortunately still increasing.


TODO: fremadrettet er der ting som kunne forbedres: LLVM, Optimizations, div. language features (data types and pattern matching being the most essential)

Looking forward, Kite still lacks a few features before being a practically useful language, namely the construction of ones own data types, a backend emitting LLVM and fully functional pattern matching. In the section \nameref{sec:discussion} we have looked at how this can be achieved.

As of now, Kite implements several features of a functional language, but still lack some features before being a fully fletched language.
