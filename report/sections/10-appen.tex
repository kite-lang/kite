%%% Local Variables:
%%% mode: latex
%%% TeX-master: "../report"
%%% End:

\subsection{Kite Foundation}

This documentation is similar, and partly based on, Haskell's
documentation of Prelude.

\subsubsection{Constants}

\texttt{pi :: Float}\ \\
The constant $\pi$ (3.14159265359)


\subsubsection{Functional}

\texttt{id :: a -> a}\ \\
Identity function


\subsubsection{Composition}

\texttt{{.} :: (b -> c) -> (a -> b) -> (a -> c)}\ \\
Function composition

\texttt{flip :: (a -> b -> c) -> (b -> a -> c)}\ \\
\texttt{flip f} takes its (first) two arguments in the reverse order of \texttt{f}.


\subsubsection{Boolean}

\texttt{not :: Bool -> Bool}\ \\
Logical NOT

\texttt{{\&\&} :: Bool -> Bool -> Bool}\ \\
Logical AND

\texttt{{||} :: Bool -> Bool -> Bool}\ \\
Logical OR

\texttt{{==} :: a -> a -> Bool}\ \\
Equality (\emph{built-in})

\texttt{{/=} :: a -> a -> Bool}\ \\
Negated equality

\texttt{{<=} :: a -> a -> Bool}\ \\
Logical less-than or equals (\emph{built-in})

\texttt{{<} :: a -> a -> Bool}\ \\
Logical less-than

\texttt{{>} :: a -> a -> Bool}\ \\
Logical greater-than

\texttt{{>=} :: a -> a -> Bool}\ \\
Logical greater-than or equals


\subsubsection{Pairs}

\texttt{asPair :: a -> b -> (a, b)}\ \\
Pairs two first arguments

\texttt{{<>} :: a -> b -> (a, b)}\ \\
Infix operator for \texttt{asPair}

\texttt{fst :: (a, b) -> a}\ \\
First element of pair

\texttt{snd :: (a, b) -> b}\ \\
Second element of pair


\subsubsection{Lists}

\texttt{{\#} :: [a] -> Int -> a}\ \\
List index

\texttt{length :: [a] -> Int}\ \\
Length of list

\texttt{empty :: [a] -> Bool}\ \\
Test for empty list

\texttt{head :: [a] -> a}\ \\
First element of list

\texttt{tail :: [a] -> [a]}\ \\
Elements of list, \emph{after} head

\texttt{init :: [a] -> [a]}\ \\\ \\
ements of list, \emph{except} head

\texttt{last :: [a] -> a}\ \\
Last element of list

\texttt{{++} :: [a] -> [a] -> [a]}\ \\
Join two lists

\texttt{map :: (a -> b) -> [a] -> [b]}\ \\
\texttt{map f xs} is the list of results, of applying \texttt{f} to each element in \texttt{xs}

\texttt{fold :: (a -> b -> a) -> a -> [b] -> a}\ \\
\texttt{fold}, applied to a binary operator, a starting value and a list,
   reduces the list using the binary operator, from left to right

\texttt{fold1 :: (a -> b -> a) -> [b] -> a}\ \\
\texttt{fold1}, applied to a binary operator and a list, reduces the list
   using the binary operator, from left to right, with head as
   starting value

\texttt{filter :: (a -> Bool) -> [a] -> [a]}\ \\
\texttt{filter f xs} returns the list of results which holds for the
   predicate function \texttt{f}, where \texttt{f} is applied to each element in \texttt{xs}

\texttt{take :: Int -> [a] -> [a]}\ \\
\texttt{take n xs} returns the \texttt{n} first elements in \texttt{xs}

\texttt{drop :: Int -> [a] -> [a]}\ \\\ \\
\texttt{drop n xs} returns the elements of \texttt{xs}, \emph{after} the \texttt{n} first elements

\texttt{zip :: [a] -> [b] -> [(a, b)]}\ \\
\texttt{zip} takes two lists and returns a list of corresponding pairs. If
   one input list is short, excess elements of the longer list are
   discarded

\texttt{zipWith :: (a -> b -> c) -> [a] -> [b] -> [c]}\ \\
\texttt{zipWith} takes a function and two lists, and returns a the
   function applied to an element of the first list and the
   corresponding element of the second list. If one input list is
   short, excess elements of the longer list are discarded

\texttt{and :: [Bool] -> Bool}\ \\
Conjunction of Boolean list

\texttt{or :: [Bool] -> Bool}\ \\
Disjunction of Boolean list

\texttt{all :: (a -> Bool) -> [a] -> Bool}\ \\
\texttt{all f xs} tests if all elements in \texttt{xs} hold for predicate
   functions \texttt{f}, where \texttt{f} is applied to each element

\texttt{any :: (a -> Bool) -> [a] -> Bool}\ \\
\texttt{any f xs} tests if any of the elements in \texttt{xs} hold for predicate
   functions \texttt{f}, where \texttt{f} is applied to each element

\texttt{flatten :: [[a]] -> [a]}\ \\
Concatenates all lists into single list

\texttt{flatMap :: ([a] -> [b]) -> [[a]] -> [b]}\ \\
\texttt{flatMap f xs} applies each list in \texttt{xs} to function \texttt{f} and
   flattens the result into single list

\texttt{elem :: a -> [a] -> Bool}\ \\
\texttt{elem x xs} tests if element \texttt{x} is in \texttt{xs}

\texttt{reverse :: [a] -> [a]}\ \\
Returns list in reverse order

\texttt{union ::  [a] -> [a] -> [a]}\ \\
Returns the list of all distinct elements in both lists ($\cup$ in set-theory)

\texttt{intersect ::  [a] -> [a] -> [a]}\ \\
Returns the list of elements which exists in both lists ($\cap$ in set-theory)

\texttt{sortBy :: (a -> a -> Bool) -> [a] -> [a]}\ \\
Sort list by predicate function

\texttt{sort :: [a] -> [a]}\ \\
Sort list with quick-sort algorithm


\subsubsection{utilities}

\texttt{range :: Int -> Int -> [Int]}\ \\
\texttt{range x y} returns a list of Integer numbers ranging from \texttt{x} to \texttt{y}

\texttt{rangeStep :: Int -> Int -> Int -> [Int]}\ \\
\texttt{rangeStep x y z} returns a list of Integer numbers ranging from
   \texttt{x}, \texttt{x + z}, \texttt{x + 2z} up to \texttt{y}


\subsubsection{Numeric}

\texttt{{**} :: Int -> Int -> Int}\ \\
Exponentiation

\texttt{pow :: Int -> Int -> Int -> Int}\ \\
Exponentiation with base

\texttt{sum :: [a] -> a}\ \\
Sum of elements in list

\texttt{product :: [a] -> a}\ \\
Product of elements in list

\texttt{min :: a -> a -> a}\ \\
Smallest of two arguments

\texttt{max :: a -> a -> a}\ \\
Largest of two arguments

\texttt{minimum :: [a] -> a}\ \\
Smallest element in list

\texttt{maximum :: [a] -> a}\ \\
Largest element in list

\texttt{even :: Int -> Bool}\ \\
Test if number is even

\texttt{odd :: Int -> Bool}\ \\
Test if number is odd

\texttt{negate :: Int -> Int}\ \\
Negate Integer number

\texttt{negatef :: Float -> Float}\ \\
Negate Float number

\texttt{abs :: Int -> Int}\ \\
Absolute value of Integer number

\subsubsection{Misc}

\texttt{while :: (a -> Bool) -> (a -> a) -> a -> a}\ \\
\texttt{while p f x} applies \texttt{f} to \texttt{x}, while \texttt{p x} is True

\texttt{until :: (a -> Bool) -> (a -> a) -> a -> a}\ \\
\texttt{until p f x} applies \texttt{f} to \texttt{x}, until \texttt{p x} becomes False
